\documentclass[a4paper]{article}
\usepackage{tcolorbox}
\tcbuselibrary{skins}

\title{
\vspace{-3em}
\begin{tcolorbox}
    \begin{center}
      \LARGE\sffamily
      Cornell Notes on Something
    \end{center}
  \end{tcolorbox}
\vspace{-3em}
}

\date{}

\usepackage{background}
\SetBgScale{1}
\SetBgAngle{0}
\SetBgColor{red}
\SetBgContents{\rule[0em]{4pt}{\textheight}}
\SetBgHshift{-2.3cm}
\SetBgVshift{0cm}
\usepackage{lipsum}% just to generate filler text for the example
\usepackage[margin=2cm]{geometry}
\usepackage{lipsum}% just to generate dummy text for the example


%\url{http://tex.stackexchange.com/a/314/86}

\makeatletter
\def\cornell{\@ifnextchar[{\@with}{\@without}}
\def\@with[#1]#2#3{
\begin{tcolorbox}[enhanced,colback=gray,colframe=black,fonttitle=\large\bfseries\sffamily,sidebyside=true, nobeforeafter,before=\vfil,after=\vfil,colupper=blue,sidebyside align=top, lefthand width=.3\textwidth,
opacityframe=0,opacityback=.3,opacitybacktitle=1, opacitytext=1,
segmentation style={black!55,solid,opacity=0,line width=3pt},
title=#1
]
\begin{tcolorbox}[colback=red!05,colframe=red!25,sidebyside align=top,
width=\textwidth,nobeforeafter]#2\end{tcolorbox}%
\tcblower
\sffamily
\begin{tcolorbox}[colback=blue!05,colframe=blue!10,width=\textwidth,nobeforeafter]
#3
\end{tcolorbox}
\end{tcolorbox}
}
\def\@without#1#2{
\begin{tcolorbox}[enhanced,colback=white!15,colframe=white,fonttitle=\bfseries,sidebyside=true, nobeforeafter,before=\vfil,after=\vfil,colupper=blue,sidebyside align=top, lefthand width=.3\textwidth,
opacityframe=0,opacityback=0,opacitybacktitle=0, opacitytext=1,
segmentation style={black!55,solid,opacity=0,line width=3pt}
]

\begin{tcolorbox}[colback=red!05,colframe=red!25,sidebyside align=top,
width=\textwidth,nobeforeafter]#1\end{tcolorbox}%
\tcblower
\sffamily
\begin{tcolorbox}[colback=blue!05,colframe=blue!10,width=\textwidth,nobeforeafter]
#2
\end{tcolorbox}
\end{tcolorbox}
}
\makeatother

\parindent=0pt

%\newcommand{\cornell}[2]

%\AddEverypageHook{
%\hspace{.3\textwidth}\vrule width 3pt depth .4\textheight
%\vspace{-\textheight}}

\providecommand{\LyX}{L\kern-.1667em\lower.25em\hbox{Y}\kern-.125emX\@}

\begin{document}
\maketitle
\SetBgContents{\rule[0em]{4pt}{\textheight}}

\cornell{Around what time did the concept and early ETOPS rules start to be used?}{Around 1986, with more dual engine airplanes equipped with more reliable engines (B767) being used in remote routes away from suitable alternates, where 3 and 4 engined airplanes were used without any restriction to maximum alternate distance.}

\cornell{Is the term ETOPS correct? What does it stand for?}{The term ETOPS is correct. It initially stood for Extended Range Twin-Engine Operations and was widely used in the early decades. Since 2012, Annex 6 introduced the new concept of Extended Diversion Time Operations - EDTO, since having two engines was not the only parameter used for determining alternate distance limits. EDTO was not widely adopted by any regulating agencies or operators and Annex 6 itself explicitly expresses that the usage of ETOPS is acceptable. FAA has opted to reutilize the acronym to ExTended range OPerationS, and EASA uses ETOPS for the classic 2-engine case and LROPS (Long Range Operations) for 3 and 4 engines.}

\cornell{Where is the ETOPS / EDTO information and guidance?}{ICAO Annex 6, Airplane Operation, Chapter 4, and Attachment C.}

\cornell{Another research question very long for one line.}{\lipsum*[1]}

\cornell{Do you want fill much more than a page}{No problem. \lipsum*[3]}

\cornell{More?}{No problem. \lipsum*[3-4]}

\cornell[Last question]{This is the end?}{Yes.}




\end{document}
