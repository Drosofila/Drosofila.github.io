\documentclass{article}
\usepackage[margin=0cm, top=0cm]{geometry} % Top margin increased by 3cm
\usepackage{tikz}
\usepackage[absolute,overlay]{textpos}
\usepackage{lipsum} % For generating dummy text
\usepackage{setspace}
\renewcommand{\baselinestretch}{1.1}
\usepackage{graphicx}
\usepackage{amsmath}


\makeatletter
\newcommand*\bigcdot{\mathpalette\bigcdot@{.5}}
\newcommand*\bigcdot@[2]{\mathbin{\vcenter{\hbox{\scalebox{#2}{$\m@th#1\bullet$}}}}}
\makeatother

\setlength\parindent{0pt}
\begin{document}
%============================================================
% 1° Página do meu template
% Definição das linhas divisórias do meu template
%============================================================

\begin{tikzpicture}[remember picture, overlay]
    % Line closer to the upper margin
    \draw ([yshift=-2.5cm]current page.north west) -- ([yshift=-2.5cm]current page.north east);

    % Node for the first section
    \node[anchor=south west, inner sep=0, xshift=1cm, yshift=-2.5cm] (section1) {};
\end{tikzpicture}

\begin{tikzpicture}[remember picture, overlay]
    % Line closer to the upper margin
    \draw ([yshift=-4.7cm]current page.north west) -- ([yshift=-4.7cm]current page.north east);

    % Node for the second section
    \node[anchor=south west, inner sep=0, xshift=1cm, yshift=-4.7cm] (section2) {};
\end{tikzpicture}

\begin{tikzpicture}[remember picture,overlay]
  % Vertical line
  \draw ([yshift=-4.7cm,xshift=5cm]current page.north west) -- ([yshift=-\paperheight,xshift=5cm]current page.north west);

  % Node for the third section
  \node[anchor=south west, inner sep=0, xshift=6cm, yshift=-4.7cm] (section3) {};
\end{tikzpicture}

%-------------------------------------------------------------

% INFORMAÇÕES DA MATÉRIA
\begin{textblock*}{20 cm}(0.5cm,1cm) % {block width} (coords)
    \textbf{Nome da Matéria}
    \\
    \textbf{Bibliografia:} Bla bla bla
\end{textblock*}

% CÓDIGO DO RESUMO
\begin{textblock*}{3 cm}(17cm,1cm) % {block width} (coords)
    \textbf{COD}
\end{textblock*}

% CÁPITULO DO LIVRO
\begin{textblock*}{3 cm}(17cm,1.5cm) % {block width} (coords)
    \textbf{Capítulo}
\end{textblock*}

%TÓPICOS CHAVES
\begin{textblock*}{20cm}(0.5cm,3.5cm) % {block width} (coords)
   \textit{Tópicos}
\end{textblock*}

% ============================================================
% ============================================================
% RESUMO
% ============================================================
% ============================================================

% ESQUERDA
\begin{textblock*}{4cm}(0.5cm,5.5cm) % {block width} (coords)
  \textbf{Skybrary General Reading}
  \\
  \\
  \\
  Around what time did the concept and early ETOPS rules start to be used?
  \\
  \\
  Is the term ETOPS correct? What does it stand for?
  \\
  \\
  \\
  \\
  \\
  \\
  Where is the ETOPS / \\ EDTO information and guidance?
  \\
  \\
\end{textblock*}

% DIREITA
\begin{textblock*}{16cm}(05.25cm,5.5cm) % {block width} (coords)
  %\hfill \break
  test
  \\
  \\
  \\
  \\
  $\bigcdot$ Around 1986, with more dual engine airplanes equipped with more reliable engines (B767) being used in remote routes away from suitable alternates, where 3 and 4 engined airplanes were used without any restriction to maximum alternate distance.
  \\
  \\
  $\bigcdot$ The term ETOPS is correct. It initially stood for Extended Range Twin-Engine Operations and was widely used in the early decades. Since 2012, Annex 6 introduced the new concept of Extended Diversion Time Operations - EDTO, since having two engines was not the only parameter used for determining alternate distance limits. EDTO was not widely adopted by any regulating agencies or operators and Annex 6 itself explicitly expresses that the usage of ETOPS is acceptable. FAA has opted to reutilize the acronym to ExTended range OPerationS, and EASA uses ETOPS for the classic 2-engine case and LROPS (Long Range Operations) for 3 and 4 engines.
  \\
  \\
  $\bigcdot$ ICAO Annex 6, Airplane Operation, Chapter 4, and Attachment C.
\end{textblock*}


\newpage

%============================================================
% 2° Página do meu template
% Definição das linhas divisórias do meu template
%============================================================

\begin{tikzpicture}[remember picture,overlay]
 % Vertical line
   \draw ([yshift=0cm,xshift=5cm]current page.north west) -- ([yshift=-\paperheight,xshift=5cm]current page.north west);

  % Node for the third section
   \node[anchor=south west, inner sep=0, xshift=6cm, yshift=-4.7cm] (section3) {};
\end{tikzpicture}

%-------------------------------------------------------------


% ESQUERDA
\begin{textblock*}{4cm}(0.5cm,1cm) % {block width} (coords)
  Pergunta
  \\
  \\
  Pergunta
\end{textblock*}

% DIREITA
\begin{textblock*}{16cm}(05.25cm,1cm) % {block width} (coords)
  $\bigcdot$ Resposta
  \\
  \\
  $\bigcdot$ Resposta
\end{textblock*}




\end{document}
