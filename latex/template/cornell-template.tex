\documentclass{article}
\usepackage[margin=0cm, top=0cm]{geometry} % Top margin increased by 3cm
\usepackage{tikz}
\usepackage[absolute,overlay]{textpos}
\usepackage{lipsum} % For generating dummy text
\usepackage{setspace}
\renewcommand{\baselinestretch}{1.1}
\usepackage{graphicx}
\usepackage{amsmath}


\makeatletter
\newcommand*\bigcdot{\mathpalette\bigcdot@{.5}}
\newcommand*\bigcdot@[2]{\mathbin{\vcenter{\hbox{\scalebox{#2}{$\m@th#1\bullet$}}}}}
\makeatother

\setlength\parindent{0pt}
\begin{document}
%============================================================
% 1° Página do meu template
% Definição das linhas divisórias do meu template
%============================================================

\begin{tikzpicture}[remember picture, overlay]
    % Line closer to the upper margin
    \draw ([yshift=-2.5cm]current page.north west) -- ([yshift=-2.5cm]current page.north east);

    % Node for the first section
    \node[anchor=south west, inner sep=0, xshift=1cm, yshift=-2.5cm] (section1) {};
\end{tikzpicture}

\begin{tikzpicture}[remember picture, overlay]
    % Line closer to the upper margin
    \draw ([yshift=-4.7cm]current page.north west) -- ([yshift=-4.7cm]current page.north east);

    % Node for the second section
    \node[anchor=south west, inner sep=0, xshift=1cm, yshift=-4.7cm] (section2) {};
\end{tikzpicture}

\begin{tikzpicture}[remember picture,overlay]
  % Vertical line
  \draw ([yshift=-4.7cm,xshift=5cm]current page.north west) -- ([yshift=-\paperheight,xshift=5cm]current page.north west);

  % Node for the third section
  \node[anchor=south west, inner sep=0, xshift=6cm, yshift=-4.7cm] (section3) {};
\end{tikzpicture}

%-------------------------------------------------------------

% INFORMAÇÕES DA MATÉRIA
\begin{textblock*}{20 cm}(0.5cm,1cm) % {block width} (coords)
    \textbf{Nome da Matéria}
    \\
    \textbf{Bibliografia:} Bla bla bla
\end{textblock*}

% CÓDIGO DO RESUMO
\begin{textblock*}{3 cm}(17cm,1cm) % {block width} (coords)
    \textbf{COD}
\end{textblock*}

% CÁPITULO DO LIVRO
\begin{textblock*}{3 cm}(17cm,1.5cm) % {block width} (coords)
    \textbf{Capítulo}
\end{textblock*}

%TÓPICOS CHAVES
\begin{textblock*}{20cm}(0.5cm,3.5cm) % {block width} (coords)
   \textit{Tópicos}
\end{textblock*}

% ESQUERDA
\begin{textblock*}{4cm}(0.5cm,5.5cm) % {block width} (coords)
   Lorem Ipsum?
   \\

   Ipsum Lorem

\end{textblock*}

% DIREITA
\begin{textblock*}{16cm}(05.25cm,5.5cm) % {block width} (coords)
   $\bigcdot$ Lorem Ipsum: lorem ipsum lorem ipsum ipsum lorem
\\
\\
   $\bigcdot$ Lorem Ipsum: lorem ipsum lorem ipsum ipsum lorem

\end{textblock*}


\newpage

%============================================================
% 2° Página do meu template
% Definição das linhas divisórias do meu template
%============================================================

\begin{tikzpicture}[remember picture,overlay]
 % Vertical line
   \draw ([yshift=0cm,xshift=5cm]current page.north west) -- ([yshift=-\paperheight,xshift=5cm]current page.north west);

  % Node for the third section
   \node[anchor=south west, inner sep=0, xshift=6cm, yshift=-4.7cm] (section3) {};
\end{tikzpicture}

%-------------------------------------------------------------


% ESQUERDA
\begin{textblock*}{4cm}(0.5cm,1cm) % {block width} (coords)
   Lorem Ipsum?
\end{textblock*}

% DIREITA
\begin{textblock*}{16cm}(05.25cm,1cm) % {block width} (coords)
   $\bigcdot$ Lorem Ipsum





\end{textblock*}




\end{document}
